\documentclass{article}
\usepackage[utf8]{inputenc}
\usepackage{amsmath}
\usepackage{graphicx}  % Aggiungi questo pacchetto per gestire le immagini

\title{RSI Mean Reverting Strategy}
\author{}
\date{}

\begin{document}

\maketitle

Nel file \texttt{RSI\_mean\_reverting.py} viene analizzata la media dell'indicatore Relative Strength Index (RSI). L'RSI è un indicatore utilizzato per misurare la "forza" o la "debolezza" di un asset in relazione ai suoi movimenti di prezzo recenti. Esso valuta la velocità e l'intensità delle variazioni di prezzo, fornendo segnali che possono suggerire condizioni di ipercomprato o ipervenduto. In particolare, l'RSI è comunemente calcolato su un periodo di 14 giorni, ma può essere adattato ad altre durate in base alle esigenze dell'analista.

Nel contesto di questo script, l'RSI viene calcolato su un periodo, e successivamente viene applicata una media mobile semplice (SMA) all'RSI stesso. La SMA è un tipo di media che calcola la media aritmetica dei valori dell'RSI su un determinato numero di periodi, filtrando così le fluttuazioni a breve termine e aiutando a identificare tendenze più stabili.

Il sistema di trading proposto nel codice è basato sull'osservazione di eventuali inversioni della media dell'RSI. Quando viene rilevata una deviazione significativa o una inversione di tendenza rispetto alla media, il modello attiva un segnale di acquisto per l'asset, con una durata massima dell'operazione di una settimana.

A seguito di una prima analisi, i risultati ottenuti indicavano buone possibilità di battere il benchmark. Pertanto, ho deciso di eseguire una serie di test aggiuntivi per verificare la robustezza e la validità del modello, cercando di evitare il rischio di overfitting \hspace{1em} \hspace{1em}.

\vspace{2em}  % Aggiungi un grande spazio verticale tra la fine della frase e la sezione successiva

\textbf{Parametri di test:}
\begin{itemize}
    \item Budget iniziale = 100
    \item Risk-free rate = 0.03
    \item Alpha = 0.05
    \item Inizio test = 2007
    \item Fine test = 2022
\end{itemize}

\textbf{Risultati del benchmark (Nasdaq 100):}
\begin{itemize}
    \item Budget finale Benchmark = 610
    \item Volatilità Benchmark = 0.02860
    \item Sharpe Ratio Benchmark = 0.05518
    \item VaR Benchmark = 0.04456
    \item Max Drawdown Benchmark = 0.51907
\end{itemize}

\vspace{2em}

% Aggiungi l'immagine "grafico benchmark"
\begin{figure}[h]
    \centering
    \includegraphics[width=0.8\textwidth]{grafico_benchmark.png}  % Sostituisci con il nome e percorso del tuo file immagine
    \caption{Grafico Benchmark}  % Titolo dell'immagine
    \label{fig:grafico_benchmark}
\end{figure}

\vspace{2em}  % Distanza tra il grafico e la sezione successiva

% Aggiungi la sezione dei risultati della strategia
\textbf{Risultati della Strategia:}
\begin{itemize}
    \item Budget finale Strategia = 610
    \item Volatilità Strategia = 0.02860
    \item Sharpe Ratio Strategia = 0.05518
    \item VaR Strategia = 0.04456
    \item Max Drawdown Strategia = 0.51907
\end{itemize}

\end{document}
